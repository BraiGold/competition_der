
% Alejandro

\emph{El listado de inscriptos en cada categoría par el armado de llaves.}
\begin{lstlisting}[language=SQL]
SELECT c.idCategoria , eid.numCertificado
FROM ( (
  esIntegranteDe eid INNER JOIN Inscripcion i
  ON eid.idInscripcion = i.idInscripcion)

  INNER JOIN  esEn ee
  ON ee.idInscripcion = i.idInscripcion)

  INNER JOIN Competencia c
  ON c.idCompetencia = ee.idCompetencia

ORDER BY eid.numCertificado;
\end{lstlisting}

\emph{El país que obtuvo mayor cantidad de medallas de oro, plata y bronce.}
\begin{lstlisting}[language=SQL]
SELECT mas.idPais , COUNT(*) amount
FROM ( ( ( (
    (esEn ee INNER JOIN Inscripcion i
        ON ee.idInscripcion = i.idInscripcion)
    INNER JOIN  esIntegranteDe eid
        ON eid.idInscripcion = i.idInscripcion )
    INNER JOIN Estudiante e
        ON e.numCertificado = eid.numCertificado)
    INNER JOIN Escuela esc
        ON e.idEscuela = esc.idEscuela)
    INNER JOIN Maestro mas
        ON esc.idMaestro = mas.placaInstructor)
WHERE ee.puesto = 1 OR ee.puesto == 2 OR ee.puesto = 3
GROUP BY mas.idPais
ORDER BY amount ASC
LIMIT 1;
\end{lstlisting}


\emph{Sabiendo que las medallas de oro suman 3 puntos, las de plata 2 y las de bronce 1 punto, se quiere realizar un ranking de puntaje por país y otro por escuela.}
\begin{lstlisting}[language=SQL]
SELECT idPais , SUM(puntaje) puntajeTotal
FROM
    (SELECT mas.idPais , 3*COUNT(*) puntaje
    FROM ( ( ( (
        (esEn ee INNER JOIN Inscripcion i
            ON ee.idInscripcion = i.idInscripcion)
        INNER JOIN  esIntegranteDe eid
            ON eid.idInscripcion = i.idInscripcion )
        INNER JOIN Estudiante e
            ON e.numCertificado = eid.numCertificado)
        INNER JOIN Escuela esc
            ON e.idEscuela = esc.idEscuela)
        INNER JOIN Maestro mas
            ON esc.idMaestro = mas.placaInstructor)
    WHERE ee.puesto = 1
    GROUP BY mas.idPais

  UNION ALL

    SELECT mas.idPais , 2*COUNT(*) puntaje
    FROM ( ( ( (
        (esEn ee INNER JOIN Inscripcion i
            ON ee.idInscripcion = i.idInscripcion)
        INNER JOIN  esIntegranteDe eid
            ON eid.idInscripcion = i.idInscripcion )
        INNER JOIN Estudiante e
            ON e.numCertificado = eid.numCertificado)
        INNER JOIN Escuela esc
            ON e.idEscuela = esc.idEscuela)
        INNER JOIN Maestro mas
            ON esc.idMaestro = mas.placaInstructor)
    WHERE ee.puesto = 2
    GROUP BY mas.idPais

  UNION ALL

    SELECT mas.idPais , COUNT(*) puntaje
    FROM ( ( ( (
        (esEn ee INNER JOIN Inscripcion i
            ON ee.idInscripcion = i.idInscripcion)
        INNER JOIN  esIntegranteDe eid
            ON eid.idInscripcion = i.idInscripcion )
        INNER JOIN Estudiante e
            ON e.numCertificado = eid.numCertificado)
        INNER JOIN Escuela esc
            ON e.idEscuela = esc.idEscuela)
        INNER JOIN Maestro mas
            ON esc.idMaestro = mas.placaInstructor)
    WHERE ee.puesto = 3
    GROUP BY mas.idPais)

GROUP BY idPais
ORDER BY puntajeTotal ASC;




SELECT idEscuela  , SUM(puntaje) puntajeTotal
FROM
    (SELECT esc.idEscuela, 3 * COUNT(*) puntaje
    FROM ( ( (
        (esEn ee INNER JOIN Inscripcion i
            ON ee.idInscripcion = i.idInscripcion)
        INNER JOIN  esIntegranteDe eid
            ON eid.idInscripcion = i.idInscripcion )
        INNER JOIN Estudiante e
            ON e.numCertificado = eid.numCertificado)
        INNER JOIN Escuela esc
            ON e.idEscuela = esc.idEscuela)
    WHERE ee.puesto = 1
    GROUP BY esc.idEscuela

  UNION ALL

    SELECT esc.idEscuela , 2*COUNT(*) puntaje
    FROM ( ( (
        (esEn ee INNER JOIN Inscripcion i
            ON ee.idInscripcion = i.idInscripcion)
        INNER JOIN  esIntegranteDe eid
            ON eid.idInscripcion = i.idInscripcion )
        INNER JOIN Estudiante e
            ON e.numCertificado = eid.numCertificado)
        INNER JOIN Escuela esc
            ON e.idEscuela = esc.idEscuela)
    WHERE ee.puesto = 2
    GROUP BY esc.idEscuela

  UNION ALL

    SELECT esc.idEscuela , COUNT(*) puntaje
    FROM ( ( (
        (esEn ee INNER JOIN Inscripcion i
            ON ee.idInscripcion = i.idInscripcion)
        INNER JOIN  esIntegranteDe eid
            ON eid.idInscripcion = i.idInscripcion )
        INNER JOIN Estudiante e
            ON e.numCertificado = eid.numCertificado)
        INNER JOIN Escuela esc
            ON e.idEscuela = esc.idEscuela)
    WHERE ee.puesto = 3
    GROUP BY esc.idEscuela)

GROUP BY idEscuela
ORDER BY puntajeTotal ASC;
\end{lstlisting}


\emph{Dado un competidor, la lista de categorías donde haya participado y el resultado obtenido.}
\begin{lstlisting}[language=SQL]
SELECT comp.idCategoria , ee.puesto
FROM (
    (esIntegranteDe eid INNER JOIN inscripcion i
     ON eid.idInscripcion = i.idInscripcion )
    INNER JOIN esEn ee ON ee.idInscripcion = i.idInscripcion)
    INNER JOIN Competencia comp
        ON ee.idCompetencia = comp.idCompetencia
WHERE eid.numCertificado = @numCertificado;
\end{lstlisting}


% Brian

\emph{El medallero por escuela.}
\begin{lstlisting}[language=SQL]
Select idEscBronces idEscuela, cantOros, cantPlatas, cantBronces
from (select escuela.idEscuela idEscBronces ,
             COUNT(broncesPorEsc.idEsc) cantBronces
             from Escuela
      left join
      (Select distinct  E.idEscuela idEsc , I.idInscripcion
      from esen
      inner join inscripcion I
          on esen.idInscripcion = I.idInscripcion
      inner join esIntegranteDe esIntDe
          on esIntDe.idInscripcion = I.idInscripcion
      inner join Estudiante E
          on esIntDe.numCertificado = E.numCertificado
      where esen.puesto = 3) broncesPorEsc

 on Escuela.idEscuela= broncesPorEsc.idEsc
 group by Escuela.idEscuela) bronces

 inner join (select escuela.idEscuela idEscPlatas,
                    COUNT(platasPorEsc.idEsc) cantPlatas
  from Escuela
left join
(Select distinct  E.idEscuela idEsc , I.idInscripcion from esen
inner join inscripcion I
    on esen.idInscripcion = I.idInscripcion
inner join esIntegranteDe esIntDe
    on esIntDe.idInscripcion = I.idInscripcion
inner join Estudiante E
    on esIntDe.numCertificado = E.numCertificado
where esen.puesto = 2) platasPorEsc

 on Escuela.idEscuela= platasPorEsc.idEsc
 group by Escuela.idEscuela) platas

 on bronces.idEscBronces = platas.idEscPlatas

 inner join
 (select escuela.idEscuela idEscOros , COUNT(orosPorEsc.idEsc) cantOros
  from Escuela
left join
(Select distinct  E.idEscuela idEsc , I.idInscripcion from esen
inner join inscripcion I
    on esen.idInscripcion = I.idInscripcion
inner join esIntegranteDe esIntDe
    on esIntDe.idInscripcion = I.idInscripcion
inner join Estudiante E
    on esIntDe.numCertificado = E.numCertificado
where esen.puesto = 1) orosPorEsc

 on Escuela.idEscuela= orosPorEsc.idEsc
 group by Escuela.idEscuela) oros

 on oros.idEscOros=platas.idEscPlatas;
\end{lstlisting}


\emph{El listado de árbitros por país.}
\begin{lstlisting}[language=SQL]
select DISTINCT p.nombre pNom , A.nombre, A.apellido,
                A.numDePlaca placa
from pais p
inner join arbitro A on p.idPais = a.idPais
order by pNom;
\end{lstlisting}


\emph{La lista de árbitros que actuaron como árbitro central en las modalidades de combate.}
\begin{lstlisting}[language=SQL]
select DISTINCT   A.nombre, A.apellido , A.numDePlaca placa
from esarbitradapor eap
inner join arbitro A on eap.numDeplacaArbitro = A.numDePlaca
inner join competencia  C on C.IdCompetencia = eap.idCompetencia

where C.tipo = 'C' and eap.funcionDelArbitro = 'Central';
\end{lstlisting}


\emph{La lista de equipos por país.}
\begin{lstlisting}[language=SQL]
select DISTINCT P.nombre pNombre , Ig.nombre equipo
from pais P
inner join maestro M
    on M.idPais = P.idPais
inner join Escuela E
    on E.idMaestro = M.placaInstructor
inner join Estudiante Est
    on Est.idEscuela = E.idEscuela
inner join esintegrantede eid
    on est.numCertificado = eid.numCertificado
inner join Inscripcion I
    on I.idInscripcion = eid.idInscripcion
inner join Inscripciongrupal Ig
    on I.idInscripcion = Ig.idInscripcion

where I.grupaloIndividual ='G'
order by pNombre;
\end{lstlisting}
