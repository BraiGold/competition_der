\documentclass[hidelinks,a4paper,11pt, nofootinbib]{article}
\usepackage[left=2.5cm,right=2.5cm,top=4cm,bottom=3.5cm]{geometry}
\usepackage[spanish, es-tabla]{babel} %es-tabla es para que ponga Tabla en vez de Cuadro en el caption
\usepackage[utf8]{inputenc}
\usepackage[T1]{fontenc}
\usepackage{xspace}
\usepackage{xargs}
\usepackage{fancyhdr}
\usepackage{lastpage}
\usepackage{caratula}
\usepackage{enumitem} %Permite modificar los margenes de lsa listas
\usepackage[bottom]{footmisc}
\usepackage{amsmath}
\usepackage{amssymb}
\usepackage{algorithm}
\usepackage[noend]{algpseudocode}
\usepackage{array}
\usepackage{xcolor,colortbl}
\usepackage{amsthm}
\usepackage{listings}
\usepackage{soul}
\usepackage{graphicx}
\usepackage{sidecap}
\usepackage{amsmath}
\usepackage{wrapfig}
\usepackage{caption}
\usepackage{mathpazo}
\usepackage{booktabs,tabularx}
\usepackage{ulem}


\setlength{\parindent}{4em}
\setlength{\parskip}{1em}

%Formato de los links
\usepackage{hyperref}
\hypersetup{
  colorlinks   = true, %Colours links instead of ugly boxes
  urlcolor     = blue, %Colour for external hyperlinks
  linkcolor    = blue, %Colour of internal links
  citecolor   = red %Colour of citations
}

\usepackage{comment}
\captionsetup[table]{labelsep=space}


\setlength{\parindent}{4em}
\setlength{\parskip}{0.5em}


%%fancyhdr
\pagestyle{fancy}
\thispagestyle{fancy}
\addtolength{\headheight}{1pt}
\lhead{Bases de Datos}
\rhead{$1º$ cuatrimestre de 2017}
\cfoot{\thepage\ / \pageref{LastPage}}
\renewcommand{\footrulewidth}{0.4pt}
\renewcommand{\labelitemi}{$\bullet$}

%%caratula
\materia{Bases de Datos}
\titulo{Trabajo Práctico Número 1}
\subtitulo{Sistema de inscripción: Mundial de Irlanda 2017}
\grupo{Grupo 3}
\integrante{Ciruelos Rodríguez, Gonzalo}{063/14}{gonzalo.ciruelos@gmail.com}
\integrante{Ferrante, Alejandro}{371/09}{matapalabras@hotmail}
\integrante{Goldstein, Brian}{027/14}{brai.goldstein@gmail.com}
\integrante{Thibeault, Gabriel}{114/13}{gabriel.eric.thibeault@gmail.com}

% \fecha{24 de Junio de 2016}
\begin{document}

\maketitle

\tableofcontents
\newpage

\section{Introducción}
El trabajo práctico se basa en\ldots

\newpage

\section{Modelos y Diseños}
\subsection{Diagrama de Entidad--Relación}

El MER se encuentra en la siguiente página. Si tiene problemas para verlo, por favor use el pdf y use zoom porque la imagen es vectorial y puede agrandarse cuanto uno quiera.

\subsubsection{Suposiciones}

A la hora de hacer el diagrama de entidad--relación hicimos algunas asunciones sobre el problema, sobre todo en aspectos en los que la consigna era vaga. A continuación se encuentran los más importantes.

\begin{enumerate}
	\item Los alumnos de la escuela y el maestro son del mismo país que su escuela.
	\item Los puestos de las competencias solo van a ser 1, 2 y 3.
	\item Los certificados ITF son únicos.
	\item En toda competencia debe haber al menos 3 competidores.
	\item Puede haber categorías solapadas en una misma disciplina.
\end{enumerate}


\subsubsection{Restricciones}

Estas son las restricciones

\begin{enumerate}
  \item Para una escuela, tiene que haber un coach cada 5 competidores.
  \item Si una inscripción es grupal, debe tener 8 participantes, 5 titulares y 3 suplentes.
  \item Toda competencia tiene mínimo 3 inscriptos.
  \item Para cada competencia (luego de haberse jugado) tiene que haber exactamente un primer puesto y exactamente un segundo puesto y exactamente un tercer puesto.
  \item En cada competencia tiene que haber al menos un presidente de mesa, un arbitro central, varios jueces y 3 suplentes.
  \item El número de jueces debe ser mayor o igual que el cantidadDeArbitros de la competencia.


  \item Si una inscripción es individual debe ser integrada por un participante y coacheada por un coach.
  \item Cantidad de participantes por inscripción: \begin{enumerate}
    \item Combate por equipos tiene inscripciones grupales.
    \item El resto tiene inscripciones individuales. \end{enumerate}
  \item  Si una inscripción es individual, entonces el participante es titular.
  \item Todos los participantes y coaches de una inscripción deben ser de la misma escuela.
  \item Un participante no puede estar en 2 equipos.
  \item La disciplina de una competencia debe corresponderse a un tipo de categoría particular: \begin{enumerate}
    \item combate: peso, edad, sexo, dan.
    \item formas: sexo edad, dan.
    \item salto: sexo, edad, dan.
    \item rotura: sexo, dan
    \item combate por equipos: sexo \end{enumerate}
  \item Si un participante esta en una inscripción en una categoría, debe tener el peso correspondiente.
  \item Si un participante esta en una inscripción en una categoría, debe tener el genero correspondiente.
  \item Si un participante esta en una inscripción en una categoría, debe tener el dan correspondiente.
  \item Si un participante esta en una inscripción en una categoría, debe tener la edad correspondiente.
  \item Si un arbitro arbitra mas de una competencia, todas tienen que ser en el mismo ring.
  \item El mínimo dan de los jueces de una competencia debe ser mayor al máximo de los competidores.
  \item No puede haber competencias del mismo tipo con iguales categorías.
\end{enumerate}

\subsubsection{Diagrama}

\newpage

\includegraphics[angle=90,height=20cm]{../mer/mer.pdf}


\newpage

\subsection{Modelo Relacional}
\par País (\underline{idPaís}, nombre)

\par Maestro (\underline{númPlacaInstructor}, nombre, apellido, graduación, \dotuline{idPaís})
\par Escuela (\underline{idEscuela}, nombre, \dotuline{idMaestro})

\par Estudiante (\underline{númCertificado}, nombre, apellido, género, graduación, peso, foto, \dotuline{idEscuela})
\par Participante (\underline{\dotuline{númCertificado}}, documentoDeIdentidad, fechaDeNacimiento)
\par Coach (\underline{\dotuline{númCertificado}})

\par esIntegranteDe (\underline{\dotuline{númCertificado}}, \underline{\dotuline{idInscripción}}, esTitular)
\par Inscripción (\underline{idInscripción}, nombre, \dotuline{idCoach}, grupalOIndividual)
\par InscripciónIndividual (\underline{\dotuline{idInscripción}})
\par InscripciónGrupal (\underline{\dotuline{idInscripción}}, nombre)

\par esEn (\underline{\dotuline{idInscripción}}, \underline{\dotuline{idCompetencia}})

\par Competencia (\underline{idCompetencia}, \dotuline{idCategoría}, cantidadDeÁrbitros, tipo)
\par CompetenciaFormas (\underline{\dotuline{idCompetencia}})
\par CompetenciaCombate (\underline{\dotuline{idCompetencia}})
\par CompetenciaSalto (\underline{\dotuline{idCompetencia}})
\par CompetenciaRotura (\underline{\dotuline{idCompetencia}})
\par CompetenciaCombatePorEquipos (\underline{\dotuline{idCompetencia}})

\par Categoría (\underline{idCategoría}, género)
\par CategoríaDan (\underline{\dotuline{idCategoría}}, dan)
\par CategoríaEdad (\underline{\dotuline{idCategoría}}, mín, máx)
\par CategoríaPeso (\underline{\dotuline{idCategoría}}, mín, máx)

\par Ring (\underline{númeroDeRing})
\par Árbitro (\underline{númPlacaÁrbitro}, nombre, apellido, graduación, \dotuline{idPaís})

\par esArbitradaPor (\underline{\dotuline{idCategoría}}, \underline{\dotuline{númPlacaÁrbitro}}, \dotuline{númeroDeRing}, función)

\newpage

\subsection{Base de Datos SQLite}
A continuación se encuentra el schema de la base de datos de SQLite que creamos basándonos en el modelo relacional.

\begin{lstlisting}[language=SQL]
create table Pais (
idPais int not null,
nombre varchar(30),
Primary Key (idPais)
);

create table Maestro (
placaInstructor int not null,
nombre varchar(30),
apellido varchar(30),
graduacion int,
idPais int not null,
Primary Key (placaInstructor),
Foreign Key (idPais) References Pais(idPais)
);

create table Escuela (
idEscuela int not null,
idMaestro int not null,
Primary Key (idEscuela),
Foreign Key (idMaestro) References Maestro(placaInstructor)
);

create table Estudiante (
numCertificado int not null,
nombre varchar(30),
apellido varchar(30),
genero char(1),
graduacion int,
peso int,
foto binary(9),
idEscuela int not null,
Primary Key (numCertificado),
Foreign Key (idEscuela) References Escuela(idEscuela)
);

create table Participante (
numCertificado int not null,
DNI int,
fechaDeNacimiento int,
Primary Key (numCertificado),
Foreign Key (numCertificado) References Estudiante(numCertificado)
);

create table Coach (
numCertificado int not null,
Primary Key (numCertificado),
Foreign Key (numCertificado) References Estudiante(numCertificado)
);

create table esIntegranteDe (
numCertificado int not null,
idInscripcion int not null,
esTitular bool,
Primary Key (numCertificado, idInscripcion),
Foreign Key (numCertificado) References Estudiante(numCertificado),
Foreign Key (idInscripcion) References Inscripcion(idInscripcion)
);

create table Inscripcion (
idInscripcion int not null,
idCoach int not null,
grupalOIndividual char(1),
Primary Key (idInscripcion),
Foreign Key (idCoach) References Coach(numCertificado)
);

create table InscripcionIndividual (
idInscripcion int not null,
Primary Key (idInscripcion),
Foreign Key (idInscripcion) References Inscripcion(idInscripcion)
);

create table InscripcionGrupal (
idInscripcion int not null,
nombre varchar(30),
Primary Key (idInscripcion),
Foreign Key (idInscripcion) References Inscripcion(idInscripcion)
);

create table esEn (
idCompetencia int not null,
idInscripcion int not null,
puesto int,
Primary Key (idCompetencia, idInscripcion),
Foreign Key (idCompetencia) References Competencia(idCompetencia),
Foreign Key (idInscripcion) References Inscripcion(idInscripcion)
);

create table Competencia (
idCompetencia int not null,
idCategoria int not null,
cantidadDeArbitros int,
tipo varchar(30),
Primary Key (idCompetencia),
Foreign Key (idCategoria) References Categoria(idCategoria)
);

create table CompetenciaFormas (
idCompetencia int not null,
Primary Key (idCompetencia),
Foreign Key (idCompetencia) References Competencia(idCompetencia)
);

create table CompetenciaCombate (
idCompetencia int not null,
Primary Key (idCompetencia),
Foreign Key (idCompetencia) References Competencia(idCompetencia)
);

create table CompetenciaSalto (
idCompetencia int not null,
Primary Key (idCompetencia),
Foreign Key (idCompetencia) References Competencia(idCompetencia)
);

create table CompetenciaRotura (
idCompetencia int not null,
Primary Key (idCompetencia),
Foreign Key (idCompetencia) References Competencia(idCompetencia)
);

create table CompetenciaCombatePorEquipos (
idCompetencia int not null,
Primary Key (idCompetencia),
Foreign Key (idCompetencia) References Competencia(idCompetencia)
);

create table Categoria (
idCategoria int not null,
genero char(1),
Primary Key (idCategoria)
);

create table CategoriaDan (
idCategoria int not null,
dan int,
Primary Key (idCategoria),
Foreign Key (idCategoria) References Categoria(idCategoria)
);

create table CategoriaEdad (
idCategoria int not null,
minima int,
maxima int,
Primary Key (idCategoria),
Foreign Key (idCategoria) References Categoria(idCategoria)
);

create table CategoriaPeso (
idCategoria int not null,
minimo int,
maximo int,
Primary Key (idCategoria),
Foreign Key (idCategoria) References Categoria(idCategoria)
);

create table Ring (
numeroDeRing int not null,
Primary Key (numeroDeRing)
);

create table Arbitro (
numDePlaca int not null,
nombre varchar(30),
apellido varchar(30),
graduacion int,
idPais int not null,
Primary Key (numDePlaca),
Foreign Key (idPais) References Pais(idPais)
);

create table esArbitradaPor (
idCompetencia int not null,
numDePlacaArbitro int not null,
numeroDeRing int not null,
funcionDelArbitro varchar(30),
Primary Key (idCompetencia, numDePlacaArbitro),
Foreign Key (idCompetencia) References Competencia(idCompetencia),
Foreign Key (numDePlacaArbitro) References Arbitro(numDePlacaArbitro),
Foreign Key (numeroDeRing) References Ring(numeroDeRing)
);
\end{lstlisting}

\newpage


\section{Consultas y Stored Procedures}
Nota: cuando indicamos una variable con un arroba antes de su nombre (@variable), significa que es el parámetro de la consulta, y debe ser intercambiado para hacer todas las posibles consultas.

\subsection{Funcionalidades a implementar}

% Alejandro

\emph{El listado de inscriptos en cada categoría par el armado de llaves.}
\begin{lstlisting}[language=SQL]
SELECT c.idCategoria , eid.numCertificado 
FROM ( (
	esIntegranteDe eid INNER JOIN Inscripcion i 
	ON eid.idInscripcion = i.idInscripcion) 
	
	INNER JOIN  esEn ee 
	ON ee.idInscripcion = i.idInscripcion) 

	INNER JOIN Competencia c 
	ON c.idCompetencia = ee.idCompetencia

ORDER BY eid.numCertificado;
\end{lstlisting}

\emph{El país que obtuvo mayor cantidad de medallas de oro, plata y bronce.}
\begin{lstlisting}[language=SQL]
SELECT	mas.idPais , COUNT(*) amount 
FROM ( ( ( ( (esEn ee INNER JOIN Inscripcion i ON ee.idInscripcion = i.idInscripcion) INNER JOIN  esIntegranteDe eid ON eid.idInscripcion = i.idInscripcion ) INNER JOIN Estudiante e ON e.numCertificado = eid.numCertificado) INNER JOIN Escuela esc ON e.idEscuela = esc.idEscuela) INNER JOIN Mestro mas ON esc.idMaestro = mas.placaInstructor) 
WHERE ee.puesto = 1 OR ee.puesto == 2 OR ee.puesto = 3 
GROUP BY mas.idPais 
ORDER BY amount ASC 
LIMIT 1;
\end{lstlisting}


\emph{Sabiendo que las medallas de oro suman 3 puntos, las de plata 2 y las de bronce 1 punto, se quiere realizar un ranking de puntaje por país y otro por escuela.}
\begin{lstlisting}[language=SQL]
SELECT idPais , SUM(puntaje) puntajeTotal
FROM
		(SELECT	mas.idPais , 3*COUNT(*) puntaje 
		FROM ( ( ( ( (esEn ee INNER JOIN Inscripcion i ON ee.idInscripcion = i.idInscripcion) INNER JOIN  esIntegranteDe eid ON eid.idInscripcion = i.idInscripcion ) INNER JOIN Estudiante e ON e.numCertificado = eid.numCertificado) INNER JOIN Escuela esc ON e.idEscuela = esc.idEscuela) INNER JOIN Mestro mas ON esc.idMaestro = mas.placaInstructor) 
		WHERE ee.puesto = 1
		GROUP BY mas.idPais)

	UNION

		(SELECT	mas.idPais , 2*COUNT(*) puntaje 
		FROM ( ( ( ( (esEn ee INNER JOIN Inscripcion i ON ee.idInscripcion = i.idInscripcion) INNER JOIN  esIntegranteDe eid ON eid.idInscripcion = i.idInscripcion ) INNER JOIN Estudiante e ON e.numCertificado = eid.numCertificado) INNER JOIN Escuela esc ON e.idEscuela = esc.idEscuela) INNER JOIN Mestro mas ON esc.idMaestro = mas.placaInstructor) 
		WHERE ee.puesto = 2
		GROUP BY mas.idPais)

	UNION

		(SELECT	mas.idPais , COUNT(*) puntaje 
		FROM ( ( ( ( (esEn ee INNER JOIN Inscripcion i ON ee.idInscripcion = i.idInscripcion) INNER JOIN  esIntegranteDe eid ON eid.idInscripcion = i.idInscripcion ) INNER JOIN Estudiante e ON e.numCertificado = eid.numCertificado) INNER JOIN Escuela esc ON e.idEscuela = esc.idEscuela) INNER JOIN Mestro mas ON esc.idMaestro = mas.placaInstructor) 
		WHERE ee.puesto = 3
		GROUP BY mas.idPais)

GROUP BY idPais
ORDER BY puntajeTotal ASC;




SELECT idEscuela  , SUM(puntaje) puntajeTotal
FROM
		(SELECT	mas.idEscuela , 3*COUNT(*) puntaje 
		FROM ( ( ( ( (esEn ee INNER JOIN Inscripcion i ON ee.idInscripcion = i.idInscripcion) INNER JOIN  esIntegranteDe eid ON eid.idInscripcion = i.idInscripcion ) INNER JOIN Estudiante e ON e.numCertificado = eid.numCertificado) INNER JOIN Escuela esc ON e.idEscuela = esc.idEscuela) INNER JOIN Mestro mas ON esc.idMaestro = mas.placaInstructor) 
		WHERE ee.puesto = 1
		GROUP BY mas.idEscuela)

	UNION

		(SELECT	mas.idEscuela , 2*COUNT(*) puntaje 
		FROM ( ( ( ( (esEn ee INNER JOIN Inscripcion i ON ee.idInscripcion = i.idInscripcion) INNER JOIN  esIntegranteDe eid ON eid.idInscripcion = i.idInscripcion ) INNER JOIN Estudiante e ON e.numCertificado = eid.numCertificado) INNER JOIN Escuela esc ON e.idEscuela = esc.idEscuela) INNER JOIN Mestro mas ON esc.idMaestro = mas.placaInstructor) 
		WHERE ee.puesto = 2
		GROUP BY mas.idEscuela)

	UNION

		(SELECT	mas.idEscuela , COUNT(*) puntaje 
		FROM ( ( ( ( (esEn ee INNER JOIN Inscripcion i ON ee.idInscripcion = i.idInscripcion) INNER JOIN  esIntegranteDe eid ON eid.idInscripcion = i.idInscripcion ) INNER JOIN Estudiante e ON e.numCertificado = eid.numCertificado) INNER JOIN Escuela esc ON e.idEscuela = esc.idEscuela) INNER JOIN Mestro mas ON esc.idMaestro = mas.placaInstructor) 
		WHERE ee.puesto = 3
		GROUP BY mas.idEscuela)

GROUP BY idEscuela
ORDER BY puntajeTotal ASC;
\end{lstlisting}


\emph{Dado un competidor, la lista de categorías donde haya participado y el resultado obtenido.}
\begin{lstlisting}[language=SQL]
SELECT	comp.idCategoria , ee.puesto
FROM ( ( esIntegranteDe eid INNER JOIN inscripcion i ON eid.idInscripcion = i.idInscripcion ) INNER JOIN esEn ee ON ee.idInscripcion = i.idInscripcion) INNER JOIN Competencia comp ON ee.idCompetencia = comp.idCompetencia 
WHERE eid.numCertificado =: @numcert;
\end{lstlisting}


% Brian

\emph{El medallero por escuela.}
\begin{lstlisting}[language=SQL]
%% ACA LA QUERY
\end{lstlisting}


\emph{El listado de árbitros por país.}
\begin{lstlisting}[language=SQL]
%% ACA LA QUERY
\end{lstlisting}


\emph{La lista de árbitros que actuaron como árbitro central en las modalidades de combate.}
\begin{lstlisting}[language=SQL]
%% ACA LA QUERY
\end{lstlisting}


\emph{La lista de equipos por país.}
\begin{lstlisting}[language=SQL]
%% ACA LA QUERY
\end{lstlisting}

\subsection{Restricciones}


Para una escuela, tiene que haber un coach cada 5 competidores.

\begin{lstlisting}[language=SQL]
select (
    select count(p.numcertificado)
    from participante p, estudiante e
    where p.numcertificado = e.numcertificado and e.idescuela = @idEscuela)
  <= 5* (
    select count(c.numcertificado)
    from coach c, estudiante e
    where c.numcertificado = e.numcertificado and e.idescuela = @idEscuela);
\end{lstlisting}

Si una inscripción es grupal, debe tener 8 participantes, 5 titulares y 3 suplentes.

\begin{lstlisting}[language=SQL]
select (
  select count(p.numcertificado)
  from participante p, esintegrantede e
  where e.numcertificado = p.numcertificado and
              e.idinscripcion = @idInscripcionGrupal and
	      e.estitular) = 5 and (
  select count(p.numcertificado)
  from participante p, esintegrantede e
  where e.numcertificado = p.numcertificado and
              e.idinscripcion = @idInscripcionGrupal and
	      not e.estitular) = 3;
\end{lstlisting}

Toda competencia tiene mínimo 3 inscriptos.

\begin{lstlisting}[language=SQL]
select (
  select count(e.idinscripcion)
  from esen e
  where e.idcompetencia = @idCompetencia) >= 3;
\end{lstlisting}

Tiene que haber exactamente un primer puesto y exactamente un segundo puesto y exactamente un tercer puesto.

\begin{lstlisting}[language=SQL]
select (
  select count(e.idinscripcion)
  from esen e
  where e.idcompetencia = @idCompetencia and e.puesto = 1) = 1 and (
  select count(e.idinscripcion)
  from esen e
  where e.idcompetencia = @idCompetencia and e.puesto = 2) = 1 and (
  select count(e.idinscripcion)
  from esen e
  where e.idcompetencia = @idCompetencia and e.puesto = 3) = 1;
\end{lstlisting}

En cada competencia tiene que haber al menos: un presidente de mesa, un arbitro central, varios jueces y al menos 3 suplentes.

\begin{lstlisting}[language=SQL]
select (
  select count(e.numdeplacaarbitro)
  from esarbitradapor e
  where e.idcompetencia = @idCompetencia and
        e.funciondelarbitro = "PresidenteDeMesa") >= 1 and (
  select count(e.numdeplacaarbitro)
  from esarbitradapor e
  where e.idcompetencia = @idCompetencia and
        e.funciondelarbitro = "Central") >= 1 and (
  select count(e.numdeplacaarbitro)
  from esarbitradapor e
  where e.idcompetencia = @idCompetencia and
        e.funciondelarbitro = "Juez") >= 2 and (
  select count(e.numdeplacaarbitro)
  from esarbitradapor e
  where e.idcompetencia = @idCompetencia and
        e.funciondelarbitro = "Suplente") >= 3;
\end{lstlisting}

El numero de jueces debe coincidir con cantidadDeArbitros de la competencia.

\begin{lstlisting}[language=SQL]
select (
  select count(e.numdeplacaarbitro)
  from esarbitradapor e
  where e.idcompetencia = @idCompetencia) = (
  select c.cantidaddearbitros
  from competencia c
  where c.idcompetencia = @idCompetencia);
\end{lstlisting}

\newpage


\section{Conclusión}
\input{conclusion.tex}
\newpage


%\begin{thebibliography}{9}
%
%\bibitem{inpractice}
%  Len Bass et al.,
%  \emph{Software Architecture in Pratice}.
%  Addison-Wesley,
%  2nd edition.
%
%\end{thebibliography}


\end{document}
