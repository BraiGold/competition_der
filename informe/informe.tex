\documentclass[hidelinks,a4paper,11pt, nofootinbib]{article}
\usepackage[left=2.5cm,right=2.5cm,top=4cm,bottom=3.5cm]{geometry}
\usepackage[spanish, es-tabla]{babel} %es-tabla es para que ponga Tabla en vez de Cuadro en el caption
\usepackage[utf8]{inputenc}
\usepackage[T1]{fontenc}
\usepackage{xspace}
\usepackage{xargs}
\usepackage{fancyhdr}
\usepackage{lastpage}
\usepackage{caratula}
\usepackage{enumitem} %Permite modificar los margenes de lsa listas
\usepackage[bottom]{footmisc}
\usepackage{amsmath}
\usepackage{amssymb}
\usepackage{algorithm}
\usepackage[noend]{algpseudocode}
\usepackage{array}
\usepackage{xcolor,colortbl}
\usepackage{amsthm}
\usepackage{listings}
\usepackage{soul}
\usepackage{graphicx}
\usepackage{sidecap}
\usepackage{amsmath}
\usepackage{wrapfig}
\usepackage{caption}
\usepackage{mathpazo}
\usepackage{booktabs,tabularx}
\usepackage{ulem}


\setlength{\parindent}{4em}
\setlength{\parskip}{1em}

%Formato de los links
\usepackage{hyperref}
\hypersetup{
  colorlinks   = true, %Colours links instead of ugly boxes
  urlcolor     = blue, %Colour for external hyperlinks
  linkcolor    = blue, %Colour of internal links
  citecolor   = red %Colour of citations
}

\usepackage{comment}
\captionsetup[table]{labelsep=space}


\setlength{\parindent}{4em}
\setlength{\parskip}{0.5em}


%%fancyhdr
\pagestyle{fancy}
\thispagestyle{fancy}
\addtolength{\headheight}{1pt}
\lhead{Bases de Datos}
\rhead{$1º$ cuatrimestre de 2017}
\cfoot{\thepage\ / \pageref{LastPage}}
\renewcommand{\footrulewidth}{0.4pt}
\renewcommand{\labelitemi}{$\bullet$}

%%caratula
\materia{Bases de Datos}
\titulo{Trabajo Práctico Número 1}
\subtitulo{Sistema de inscripción: Mundial de Irlanda 2017}
\grupo{Grupo 3}
\integrante{Ciruelos Rodríguez, Gonzalo}{063/14}{gonzalo.ciruelos@gmail.com}
\integrante{Ferrante, Alejandro}{371/09}{matapalabras@hotmail}
\integrante{Goldstein, Brian}{027/14}{brai.goldstein@gmail.com}
\integrante{Thibeault, Gabriel}{114/13}{gabriel.eric.thibeault@gmail.com}

% \fecha{24 de Junio de 2016}
\begin{document}

\maketitle

\tableofcontents
\newpage

\section{Introducción}
Este trabajo se basa en modelar al sistema de inscripciones al Mundial de Taekwondo de 2017 usando las herramientas de algún motor de base de datos.
Para eso, primero necesitamos modelar el problema con utilizando un diagrama de entidad--relación, que luego pasaremos a un modelo relacional y finalmente a una base de datos.

Se debe armar una base de datos a partir de un modelo relacional a partir de un modelo entidad relacional para cumplir con las queries requeridas.

El motor de base de datos que utilizaremos es \textbf{SQLite}. SQLite es un RDBMS que, a diferencia de muchos sistemas, no sigue el modelo cliente--servidor, si no que se encuentra embebido en el programa que la usa. Esto obviamente le da muchas limitaciones, pero al mismo tiempo mucha simplicidad, por lo que la creemos ideal para usarse en este trabajo.
SQLite implementa la mayor parte del estándar SQL, es ACID-compliant y soporta el uso de triggers.


\newpage

\section{Modelos}
\subsection{Modelo de Entidad--Relación}

El MER se encuentra en la siguiente página. Si tiene problemas para verlo, por favor use el pdf y use zoom porque la imagen es vectorial y puede agrandarse cuanto uno quiera.

\subsubsection{Suposiciones}

A la hora de hacer el diagrama de entidad--relación hicimos algunas asunciones sobre el problema, sobre todo en aspectos en los que la consigna era vaga. A continuación se encuentran los más importantes.

\begin{enumerate}
	\item Los alumnos de la escuela y el maestro son del mismo pais que su escuela.
	\item Los puestos de las competencias solo van a ser 1, 2 y 3.
	\item Los certificados ITF son únicos.
	\item En toda competencia debe haber al menos 3 competidores.
	\item Puede haber categorias solapadas en una misma disciplina.
\end{enumerate}


\subsubsection{Restricciones}

Estas son las restricciones 

\begin{enumerate}
  \item Para una escuela, tiene que haber un coach cada 5 competidores.
  \item Si una inscripción es grupal, debe tener 8 participantes, 5 titulares y 3 suplentes.
  \item Toda competencia tiene mínimo 3 inscriptos.
  \item faltan...
\end{enumerate}

\subsubsection{Diagrama}

\newpage

\includegraphics[angle=90,height=20cm]{../mer/mer.pdf}


\newpage

\subsection{Modelo Relacional}





\newpage

\section{Consultas y Stored Procedures}
Nota: cuando indicamos una variable con un arroba antes de su nombre (@variable), significa que es el parámetro de la consulta, y debe ser intercambiado para hacer todas las posibles consultas.

\subsection{Funcionalidades a implementar}

% Alejandro

\emph{El listado de inscriptos en cada categoría par el armado de llaves.}
\begin{lstlisting}[language=SQL]
SELECT c.idCategoria , eid.numCertificado
FROM ( (
  esIntegranteDe eid INNER JOIN Inscripcion i
  ON eid.idInscripcion = i.idInscripcion)

  INNER JOIN  esEn ee
  ON ee.idInscripcion = i.idInscripcion)

  INNER JOIN Competencia c
  ON c.idCompetencia = ee.idCompetencia

ORDER BY eid.numCertificado;
\end{lstlisting}

\emph{El país que obtuvo mayor cantidad de medallas de oro, plata y bronce.}
\begin{lstlisting}[language=SQL]
SELECT mas.idPais , COUNT(*) amount
FROM ( ( ( (
    (esEn ee INNER JOIN Inscripcion i
        ON ee.idInscripcion = i.idInscripcion)
    INNER JOIN  esIntegranteDe eid
        ON eid.idInscripcion = i.idInscripcion )
    INNER JOIN Estudiante e
        ON e.numCertificado = eid.numCertificado)
    INNER JOIN Escuela esc
        ON e.idEscuela = esc.idEscuela)
    INNER JOIN Mestro mas
        ON esc.idMaestro = mas.placaInstructor)
WHERE ee.puesto = 1 OR ee.puesto == 2 OR ee.puesto = 3
GROUP BY mas.idPais
ORDER BY amount ASC
LIMIT 1;
\end{lstlisting}


\emph{Sabiendo que las medallas de oro suman 3 puntos, las de plata 2 y las de bronce 1 punto, se quiere realizar un ranking de puntaje por país y otro por escuela.}
\begin{lstlisting}[language=SQL]
SELECT idPais , SUM(puntaje) puntajeTotal
FROM
    (SELECT mas.idPais , 3*COUNT(*) puntaje
    FROM ( ( ( (
        (esEn ee INNER JOIN Inscripcion i
            ON ee.idInscripcion = i.idInscripcion)
        INNER JOIN  esIntegranteDe eid
            ON eid.idInscripcion = i.idInscripcion )
        INNER JOIN Estudiante e
            ON e.numCertificado = eid.numCertificado)
        INNER JOIN Escuela esc
            ON e.idEscuela = esc.idEscuela)
        INNER JOIN Mestro mas
            ON esc.idMaestro = mas.placaInstructor)
    WHERE ee.puesto = 1
    GROUP BY mas.idPais)

  UNION

    (SELECT mas.idPais , 2*COUNT(*) puntaje
    FROM ( ( ( (
        (esEn ee INNER JOIN Inscripcion i
            ON ee.idInscripcion = i.idInscripcion)
        INNER JOIN  esIntegranteDe eid
            ON eid.idInscripcion = i.idInscripcion )
        INNER JOIN Estudiante e
            ON e.numCertificado = eid.numCertificado)
        INNER JOIN Escuela esc
            ON e.idEscuela = esc.idEscuela)
        INNER JOIN Mestro mas
            ON esc.idMaestro = mas.placaInstructor)
    WHERE ee.puesto = 2
    GROUP BY mas.idPais)

  UNION

    (SELECT mas.idPais , COUNT(*) puntaje
    FROM ( ( ( (
        (esEn ee INNER JOIN Inscripcion i
            ON ee.idInscripcion = i.idInscripcion)
        INNER JOIN  esIntegranteDe eid
            ON eid.idInscripcion = i.idInscripcion )
        INNER JOIN Estudiante e
            ON e.numCertificado = eid.numCertificado)
        INNER JOIN Escuela esc
            ON e.idEscuela = esc.idEscuela)
        INNER JOIN Mestro mas
            ON esc.idMaestro = mas.placaInstructor)
    WHERE ee.puesto = 3
    GROUP BY mas.idPais)

GROUP BY idPais
ORDER BY puntajeTotal ASC;


SELECT idEscuela  , SUM(puntaje) puntajeTotal
FROM
    (SELECT mas.idEscuela, 3 * COUNT(*) puntaje
    FROM ( ( ( (
        (esEn ee INNER JOIN Inscripcion i
            ON ee.idInscripcion = i.idInscripcion)
        INNER JOIN  esIntegranteDe eid
            ON eid.idInscripcion = i.idInscripcion )
        INNER JOIN Estudiante e
            ON e.numCertificado = eid.numCertificado)
        INNER JOIN Escuela esc
            ON e.idEscuela = esc.idEscuela)
        INNER JOIN Mestro mas
            ON esc.idMaestro = mas.placaInstructor)
    WHERE ee.puesto = 1
    GROUP BY mas.idEscuela)

  UNION

    (SELECT mas.idEscuela , 2*COUNT(*) puntaje
    FROM ( ( ( (
        (esEn ee INNER JOIN Inscripcion i
            ON ee.idInscripcion = i.idInscripcion)
        INNER JOIN  esIntegranteDe eid
            ON eid.idInscripcion = i.idInscripcion )
        INNER JOIN Estudiante e
            ON e.numCertificado = eid.numCertificado)
        INNER JOIN Escuela esc
            ON e.idEscuela = esc.idEscuela)
        INNER JOIN Mestro mas
            ON esc.idMaestro = mas.placaInstructor)
    WHERE ee.puesto = 2
    GROUP BY mas.idEscuela)

  UNION

    (SELECT mas.idEscuela , COUNT(*) puntaje
    FROM ( ( ( (
        (esEn ee INNER JOIN Inscripcion i
            ON ee.idInscripcion = i.idInscripcion)
        INNER JOIN  esIntegranteDe eid
            ON eid.idInscripcion = i.idInscripcion )
        INNER JOIN Estudiante e
            ON e.numCertificado = eid.numCertificado)
        INNER JOIN Escuela esc
            ON e.idEscuela = esc.idEscuela)
        INNER JOIN Mestro mas
            ON esc.idMaestro = mas.placaInstructor)
    WHERE ee.puesto = 3
    GROUP BY mas.idEscuela)

GROUP BY idEscuela
ORDER BY puntajeTotal ASC;
\end{lstlisting}


\emph{Dado un competidor, la lista de categorías donde haya participado y el resultado obtenido.}
\begin{lstlisting}[language=SQL]
SELECT comp.idCategoria , ee.puesto
FROM (
    (esIntegranteDe eid INNER JOIN inscripcion i
     ON eid.idInscripcion = i.idInscripcion )
    INNER JOIN esEn ee ON ee.idInscripcion = i.idInscripcion)
    INNER JOIN Competencia comp
        ON ee.idCompetencia = comp.idCompetencia
WHERE eid.numCertificado =: @numcert;
\end{lstlisting}


% Brian

\emph{El medallero por escuela.}
\begin{lstlisting}[language=SQL]
Select idEscBronces idEscuela, cantOros, cantPlatas, cantBronces
from (select escuela.idEscuela idEscBronces ,
             COUNT(broncesPorEsc.idEsc) cantBronces
             from Escuela
      left join
      (Select distinct  E.idEscuela idEsc , I.idInscripcion
      from esen
      inner join inscripcion I
          on esen.idInscripcion = I.idInscripcion
      inner join esIntegranteDe esIntDe
          on esIntDe.idInscripcion = I.idInscripcion
      inner join Estudiante E
          on esIntDe.numCertificado = E.numCertificado
      where esen.puesto = 3) broncesPorEsc

 on Escuela.idEscuela= broncesPorEsc.idEsc
 group by Escuela.idEscuela) bronces

 inner join (select escuela.idEscuela idEscPlatas,
                    COUNT(platasPorEsc.idEsc) cantPlatas
  from Escuela
left join
(Select distinct  E.idEscuela idEsc , I.idInscripcion from esen
inner join inscripcion I
    on esen.idInscripcion = I.idInscripcion
inner join esIntegranteDe esIntDe
    on esIntDe.idInscripcion = I.idInscripcion
inner join Estudiante E
    on esIntDe.numCertificado = E.numCertificado
where esen.puesto = 2) platasPorEsc

 on Escuela.idEscuela= platasPorEsc.idEsc
 group by Escuela.idEscuela) platas

 on bronces.idEscBronces = platas.idEscPlatas

 inner join
 (select escuela.idEscuela idEscOros , COUNT(orosPorEsc.idEsc) cantOros
  from Escuela
left join
(Select distinct  E.idEscuela idEsc , I.idInscripcion from esen
inner join inscripcion I
    on esen.idInscripcion = I.idInscripcion
inner join esIntegranteDe esIntDe
    on esIntDe.idInscripcion = I.idInscripcion
inner join Estudiante E
    on esIntDe.numCertificado = E.numCertificado
where esen.puesto = 1) orosPorEsc

 on Escuela.idEscuela= orosPorEsc.idEsc
 group by Escuela.idEscuela) oros

 on oros.idEscOros=platas.idEscPlatas
\end{lstlisting}


\emph{El listado de árbitros por país.}
\begin{lstlisting}[language=SQL]
select DISTINCT p.nombre pNom , A.nombre, A.apellido,
                A.numDePlaca placa
from pais p
inner join arbitro A on p.idPais = a.idPais
order by pNom
\end{lstlisting}


\emph{La lista de árbitros que actuaron como árbitro central en las modalidades de combate.}
\begin{lstlisting}[language=SQL]
select DISTINCT   A.nombre, A.apellido , A.numDePlaca placa
from esarbitradapor eap
inner join arbitro A on eap.numDeplacaArbitro = A.numDePlaca
inner join competencia  C on C.IdCompetencia = eap.idCompetencia

where C.tipo = 'C' and eap.funcionDelArbitro = 'Central'
\end{lstlisting}


\emph{La lista de equipos por país.}
\begin{lstlisting}[language=SQL]
select DISTINCT P.nombre pNombre , Ig.nombre equipo
from pais P
inner join maestro M
    on M.idPais = P.idPais
inner join Escuela E
    on E.idMaestro = M.placaInstructor
inner join Estudiante Est
    on Est.idEscuela = E.idEscuela
inner join esintegrantede eid
    on est.numCertificado = eid.numCertificado
inner join Inscripcion I
    on I.idInscripcion = eid.idInscripcion
inner join Inscripciongrupal Ig
    on I.idInscripcion = Ig.idInscripcion

where I.grupaloIndividual ='G'
order by pNombre
\end{lstlisting}

\subsection{Restricciones}





\newpage


\section{Conclusión}





\newpage


%\begin{thebibliography}{9}
%
%\bibitem{inpractice}
%  Len Bass et al.,
%  \emph{Software Architecture in Pratice}.
%  Addison-Wesley,
%  2nd edition.
%
%\end{thebibliography}


\end{document}
