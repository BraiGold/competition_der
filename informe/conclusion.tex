


El problema planteado en el trabajo era relativamente complejo, pero utilizando todas las herramientas que vimos en la materia pudimos reducirlo a subproblemas problemas más simples, lo que nos permitió hacer un muy buen diseño de una base de datos que resuelve el problema.

Plantear el Diagrama de Entidad--Relación permite entender mucho mejor el problema, y presentarlo visualmente de una manera que es mucho más fácil de comunicar a otros. Además, al hacer el Diagrama de Entidad--Relación, ciertas ambigüedad de la consigna fueron resueltas, algunas siendo sustituidas por suposiciones.

Una vez que tuvimos nuestro Diagrama de Entidad--Relación, pasar al Modelo Relacional (y por lo tanto a la Base de Datos en SQLite) fue realmente fácil, dado que es un proceso casi algorítmico.

Además, dado que nuestro diseño es bueno, implementar las restricciones del modelo como triggers y consultas en nuestra base fue también muy fácil.
