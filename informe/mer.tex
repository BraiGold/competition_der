
\subsubsection{Suposiciones}

A la hora de hacer el diagrama de entidad--relación hicimos algunas asunciones sobre el problema, sobre todo en aspectos en los que la consigna era vaga. A continuación se encuentran las más importantes.

\begin{enumerate}
	\item Los alumnos de la escuela y el maestro son del mismo país que su escuela.
	\item Los puestos de las competencias solo van a ser 1, 2 y 3.
	\item Los números de certificados ITF son únicos.
	\item En toda competencia debe haber al menos 3 competidores.
	\item Puede haber categorías solapadas en una misma disciplina.
\end{enumerate}

\subsubsection{Diagrama}

Incluimos dos layouts distintos del mismo diagrama. El primero es más prolijo pero difícil de ver en papel (se puede hacer zoom en el pdf para verlo correctamente). El segundo es más compacto pero menos prolijo.

\newpage

\noindent
\includegraphics[angle=90,height=23cm]{../mer/mer-dot.pdf}

\noindent
\includegraphics[width=\textwidth]{../mer/mer-neato.pdf}

