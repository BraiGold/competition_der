
El MER se encuentra en la siguiente página. Si tiene problemas para verlo, por favor use el pdf y use zoom porque la imagen es vectorial y puede agrandarse cuanto uno quiera.

\subsubsection{Suposiciones}

A la hora de hacer el diagrama de entidad--relación hicimos algunas asunciones sobre el problema, sobre todo en aspectos en los que la consigna era vaga. A continuación se encuentran los más importantes.

\begin{enumerate}
	\item Los alumnos de la escuela y el maestro son del mismo país que su escuela.
	\item Los puestos de las competencias solo van a ser 1, 2 y 3.
	\item Los certificados ITF son únicos.
	\item En toda competencia debe haber al menos 3 competidores.
	\item Puede haber categorías solapadas en una misma disciplina.
\end{enumerate}


\subsubsection{Restricciones}

Estas son las restricciones

\begin{enumerate}
  \item Para una escuela, tiene que haber un coach cada 5 competidores.
  \item Si una inscripción es grupal, debe tener 8 participantes, 5 titulares y 3 suplentes.
  \item Toda competencia tiene mínimo 3 inscriptos.
  \item Para cada competencia (luego de haberse jugado) tiene que haber exactamente un primer puesto y exactamente un segundo puesto y exactamente un tercer puesto.
  \item En cada competencia tiene que haber al menos un presidente de mesa, un arbitro central, varios jueces y 3 suplentes.
  \item El número de jueces debe ser mayor o igual que el cantidadDeArbitros de la competencia.


  \item Si una inscripción es individual debe ser integrada por un participante y coacheada por un coach.
  \item Cantidad de participantes por inscripción: \begin{enumerate}
    \item Combate por equipos tiene inscripciones grupales.
    \item El resto tiene inscripciones individuales. \end{enumerate}
  \item  Si una inscripción es individual, entonces el participante es titular.
  \item Todos los participantes y coaches de una inscripción deben ser de la misma escuela.
  \item Un participante no puede estar en 2 equipos.
  \item La disciplina de una competencia debe corresponderse a un tipo de categoría particular: \begin{enumerate}
    \item combate: peso, edad, sexo, dan.
    \item formas: sexo edad, dan.
    \item salto: sexo, edad, dan.
    \item rotura: sexo, dan
    \item combate por equipos: sexo \end{enumerate}
  \item Si un participante esta en una inscripción en una categoría, debe tener el peso correspondiente.
  \item Si un participante esta en una inscripción en una categoría, debe tener el genero correspondiente.
  \item Si un participante esta en una inscripción en una categoría, debe tener el dan correspondiente.
  \item Si un participante esta en una inscripción en una categoría, debe tener la edad correspondiente.
  \item Si un arbitro arbitra mas de una competencia, todas tienen que ser en el mismo ring.
  \item El mínimo dan de los jueces de una competencia debe ser mayor al máximo de los competidores.
  \item No puede haber competencias del mismo tipo con iguales categorías.
\end{enumerate}

\subsubsection{Diagrama}

\newpage

\includegraphics[angle=90,height=20cm]{../mer/mer.pdf}

