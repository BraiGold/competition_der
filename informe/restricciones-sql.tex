

Para una escuela, tiene que haber un coach cada 5 competidores.

\begin{lstlisting}[language=SQL]
select (
    select count(p.numcertificado)
    from participante p, estudiante e
    where p.numcertificado = e.numcertificado and e.idescuela = @idEscuela)
  <= 5* (
    select count(c.numcertificado)
    from coach c, estudiante e
    where c.numcertificado = e.numcertificado and e.idescuela = @idEscuela);
\end{lstlisting}

Si una inscripción es grupal, debe tener 8 participantes, 5 titulares y 3 suplentes.

\begin{lstlisting}[language=SQL]
select (
  select count(p.numcertificado)
  from participante p, esintegrantede e
  where e.numcertificado = p.numcertificado and
              e.idinscripcion = @idInscripcionGrupal and
	      e.estitular) = 5 and (
  select count(p.numcertificado)
  from participante p, esintegrantede e
  where e.numcertificado = p.numcertificado and
              e.idinscripcion = @idInscripcionGrupal and
	      not e.estitular) = 3;
\end{lstlisting}

Toda competencia tiene mínimo 3 inscriptos.

\begin{lstlisting}[language=SQL]
select (
  select count(e.idinscripcion)
  from esen e
  where e.idcompetencia = @idCompetencia) >= 3;
\end{lstlisting}

Tiene que haber exactamente un primer puesto y exactamente un segundo puesto y exactamente un tercer puesto.

\begin{lstlisting}[language=SQL]
select (
  select count(e.idinscripcion)
  from esen e
  where e.idcompetencia = @idCompetencia and e.puesto = 1) = 1 and (
  select count(e.idinscripcion)
  from esen e
  where e.idcompetencia = @idCompetencia and e.puesto = 2) = 1 and (
  select count(e.idinscripcion)
  from esen e
  where e.idcompetencia = @idCompetencia and e.puesto = 3) = 1;
\end{lstlisting}

En cada competencia tiene que haber al menos: un presidente de mesa, un arbitro central, varios jueces y al menos 3 suplentes.

\begin{lstlisting}[language=SQL]
select (
  select count(e.numdeplacaarbitro)
  from esarbitradapor e
  where e.idcompetencia = @idCompetencia and
        e.funciondelarbitro = "PresidenteDeMesa") >= 1 and (
  select count(e.numdeplacaarbitro)
  from esarbitradapor e
  where e.idcompetencia = @idCompetencia and
        e.funciondelarbitro = "Central") >= 1 and (
  select count(e.numdeplacaarbitro)
  from esarbitradapor e
  where e.idcompetencia = @idCompetencia and
        e.funciondelarbitro = "Juez") >= 2 and (
  select count(e.numdeplacaarbitro)
  from esarbitradapor e
  where e.idcompetencia = @idCompetencia and
        e.funciondelarbitro = "Suplente") >= 3;
\end{lstlisting}

El numero de jueces debe coincidir con cantidadDeArbitros de la competencia.

\begin{lstlisting}[language=SQL]
select (
  select count(e.numdeplacaarbitro)
  from esarbitradapor e
  where e.idcompetencia = @idCompetencia) = (
  select c.cantidaddearbitros
  from competencia c
  where c.idcompetencia = @idCompetencia);
\end{lstlisting}
