\par Las restricciones previamente mencionadas fueron implementadas ya sea como queries o como triggers.
Preferimos la implementación como triggers, pero ésta no siempre es posible.
Por ejemplo, considerar los puestos en las competencias: se debe poder acceder a los inscriptos previamente a la realización de la competencia, por lo que no se puede validar que los puestos no violen las restricciones al tiempo de inserción; éstas varían a lo largo de una competencia.

\par Otro caso en el que empleamos queries en vez de triggers es para restricciones que no pueden ser validadas al insertar valores uno por uno (por ejemplo que la cantidad de integrantes de una inscripción grupal sea 8, 5 titulares y 3 suplentes).
Si bien en un sistema en producción éstas deberían realizarse mediante una sola transacción atómica (que podría ser validada mediante triggers), esto escapa al TP actual, por lo que las hacemos una por una y por ende debemos validarlas mediante queries en vez de triggers.


\subsubsection{Consultas}
\emph{Para una escuela, tiene que haber un coach cada 5 competidores.}

\begin{lstlisting}[language=SQL]
select not exists (
  select  1
  from escuela esc
  where not (
        select count(p.numcertificado)
        from participante p, estudiante e
        where p.numcertificado = e.numcertificado and
              e.idescuela = esc.idescuela)
      <= 5* (
        select count(c.numcertificado)
        from coach c, estudiante e
        where c.numcertificado = e.numcertificado and
              e.idescuela = esc.idescuela));
\end{lstlisting}

\emph{Si una inscripcion es grupal, debe tener 8 participantes, 5 titulares y 3 suplentes.}

\begin{lstlisting}[language=SQL]
select (
  select count(p.numcertificado)
  from participante p, esintegrantede e
  where e.numcertificado = p.numcertificado and
              e.idinscripcion = @idInscripcionGrupal and
	      e.estitular) = 5 and (
  select count(p.numcertificado)
  from participante p, esintegrantede e
  where e.numcertificado = p.numcertificado and
              e.idinscripcion = @idInscripcionGrupal and
	      not e.estitular) = 3;
\end{lstlisting}

\emph{Toda competencia tiene minimo 3 inscriptos.}

\begin{lstlisting}[language=SQL]
select (
  select count(e.idinscripcion)
  from esen e
  where e.idcompetencia = @idCompetencia) >= 3;
\end{lstlisting}

\emph{Para cada competencia (luego de haberse jugado) tiene que haber exactamente un primer puesto y exactamente un segundo puesto y exactamente un tercer puesto.}

\begin{lstlisting}[language=SQL]
select (
  select count(e.idinscripcion)
  from esen e
  where e.idcompetencia = @idCompetencia and e.puesto = 1) = 1 and (
  select count(e.idinscripcion)
  from esen e
  where e.idcompetencia = @idCompetencia and e.puesto = 2) = 1 and (
  select count(e.idinscripcion)
  from esen e
  where e.idcompetencia = @idCompetencia and e.puesto = 3) = 1;
\end{lstlisting}

\emph{En cada competencia tiene que haber al menos un presidente de mesa, un arbitro central, varios jueces y 3 suplentes.}

\begin{lstlisting}[language=SQL]
select (
  select count(e.numdeplacaarbitro)
  from esarbitradapor e
  where e.idcompetencia = @idCompetencia and
        e.funciondelarbitro = "PresidenteDeMesa") >= 1 and (
  select count(e.numdeplacaarbitro)
  from esarbitradapor e
  where e.idcompetencia = @idCompetencia and
        e.funciondelarbitro = "Central") >= 1 and (
  select count(e.numdeplacaarbitro)
  from esarbitradapor e
  where e.idcompetencia = @idCompetencia and
        e.funciondelarbitro = "Juez") >= 2 and (
  select count(e.numdeplacaarbitro)
  from esarbitradapor e
  where e.idcompetencia = @idCompetencia and
        e.funciondelarbitro = "Suplente") >= 3;
\end{lstlisting}

\emph{El numero de jueces debe ser mayor o igual que el cantidadDeArbitros de la competencia.}

\begin{lstlisting}[language=SQL]
select (
  select count(e.numdeplacaarbitro)
  from esarbitradapor e
  where e.idcompetencia = @idCompetencia) = (
  select c.cantidaddearbitros
  from competencia c
  where c.idcompetencia = @idCompetencia);
\end{lstlisting}

\subsubsection{Triggers}

\emph{Si una inscripcion es individual debe ser integrada por un participante.}

\begin{lstlisting}[language=SQL]
CREATE TRIGGER individualUnParticipante AFTER INSERT ON esIntegranteDe
    BEGIN
        Select Raise(
          Rollback,
          "Una inscripcion individual solo puede tener un participante.")
        Where Exists (
          select 1
          from inscripcionindividual i, esintegrantede e
          where i.idinscripcion = e.idinscripcion
          group by e.idinscripcion
          having count(e.numcertificado) != 1);
    END;
\end{lstlisting}

\emph{Cantidad de participantes por inscripcion: Combate por equipos tiene inscripciones grupales.}

\begin{lstlisting}[language=SQL]
CREATE TRIGGER combatePorEquiposAceptaGrupos AFTER INSERT ON esEn
    BEGIN
        Select Raise(Rollback, "Combate por equipos solo acepta grupos.")
        Where Exists (
          select 1
          from competenciacombateporequipos c, inscripcion i, esen e
          where c.idcompetencia =e.idcompetencia and
                      i.idinscripcion = e.idinscripcion and
                (select count(e2.numcertificado)
                 from esintegrantede e2
                 where i.idinscripcion = e2.idinscripcion) != 8);
    END;
\end{lstlisting}

\emph{Cantidad de participantes por inscripcion: El resto tiene inscripciones individuales.}

\begin{lstlisting}[language=SQL]
CREATE TRIGGER combateAceptaUnParticipante AFTER INSERT ON esEn
    BEGIN
        Select Raise(Rollback, "Combate solo acepta individuales.")
        Where Exists (
          select 1
          from competenciacombate c, inscripcion i, esen e
          where c.idcompetencia =e.idcompetencia and
                      i.idinscripcion = e.idinscripcion and
                (select count(e2.numcertificado)
                 from esintegrantede e2
                 where i.idinscripcion = e2.idinscripcion) != 1);
    END;
\end{lstlisting}

\begin{lstlisting}[language=SQL]
CREATE TRIGGER formasAceptaUnParticipante AFTER INSERT ON esEn
    BEGIN
        Select Raise(Rollback, "Formas solo acepta individuales.")
        Where Exists (
          select 1
          from competenciaformas c, inscripcion i, esen e
          where c.idcompetencia =e.idcompetencia and
                      i.idinscripcion = e.idinscripcion and
                (select count(e2.numcertificado)
                 from esintegrantede e2
                 where i.idinscripcion = e2.idinscripcion) != 1);
    END;
\end{lstlisting}

\begin{lstlisting}[language=SQL]
CREATE TRIGGER roturaAceptaUnParticipante AFTER INSERT ON esEn
    BEGIN
        Select Raise(Rollback, "Rotura solo acepta individuales.")
        Where Exists (
          select 1
          from competenciarotura c, inscripcion i, esen e
          where c.idcompetencia =e.idcompetencia and
                      i.idinscripcion = e.idinscripcion and
                (select count(e2.numcertificado)
                 from esintegrantede e2
                 where i.idinscripcion = e2.idinscripcion) != 1);
    END;
\end{lstlisting}

\begin{lstlisting}[language=SQL]
CREATE TRIGGER saltoAceptaUnParticipante AFTER INSERT ON esEn
    BEGIN
        Select Raise(Rollback, "Salto solo acepta individuales.")
        Where Exists (
          select 1
          from competenciasalto c, inscripcion i, esen e
          where c.idcompetencia =e.idcompetencia and
                      i.idinscripcion = e.idinscripcion and
                (select count(e2.numcertificado)
                 from esintegrantede e2
                 where i.idinscripcion = e2.idinscripcion) != 1);
    END;
\end{lstlisting}

\emph{Si una inscripcion es individual, entonces el participante es titular.}

\begin{lstlisting}[language=SQL]
CREATE TRIGGER individualEsTitular AFTER INSERT ON esIntegranteDe
    BEGIN
        Select Raise(Rollback,
            "En una inscripcion individual, "
            "su participante debe ser titular.")
        Where Exists (
          select 1
          from inscripcionindividual i, esintegrantede e
          where i.idinscripcion = e.idinscripcion and not e.estitular);
    END;
\end{lstlisting}

\emph{En una inscripcion el coach no puede ser ninguno de los participantes.}

\begin{lstlisting}[language=SQL]

CREATE TRIGGER noSelfCoach AFTER INSERT ON esIntegranteDe
    BEGIN
        Select Raise(Rollback, "Un participante no puede coachearse a si mismo.")
        Where Exists (Select 1 From Inscripcion i, Estudiante e
                Where new.idInscripcion = i.idInscripcion And 
                i.idCoach = e.numCertificado
            And e.numCertificado = new.numCertificado);
    END;
\end{lstlisting}
  
\emph{Todos los participantes y coaches de una inscripcion deben ser de la misma escuela.}

\begin{lstlisting}[language=SQL]
CREATE TRIGGER mismaEscuela AFTER INSERT ON esIntegranteDe
    BEGIN
        Select Raise(Rollback, "Tanto los participantes como el coach" 
        "deben ser de la misma escuela.")
        Where Exists (Select 1 From esIntegranteDe eid, Estudiante e, 
        (Select es.idEscuela id From Estudiante es
                Where es.numCertificado = new.numCertificado) esc
                Where new.idInscripcion = eid.idInscripcion 
                And e.numCertificado = eid.numCertificado
            And e.idEscuela != esc.id)
        Or Exists (Select 1 From Inscripcion i, Estudiante e, 
        (Select es.idEscuela id From Estudiante es
                Where es.numCertificado = new.numCertificado) esc
                Where new.idInscripcion = i.idInscripcion And 
                i.idCoach = e.numCertificado
            And e.idEscuela != esc.id);
    END;
\end{lstlisting}

\emph{Un participante no puede estar en 2 equipos.}

\begin{lstlisting}[language=SQL]
CREATE TRIGGER unSoloTeam AFTER INSERT ON esIntegranteDe
    BEGIN
        Select Raise(Rollback, "Un participante no puede"
        "estar en mas de un equipo.")
        Where Exists (Select 1 From Inscripcion i 
        Where i.idInscripcion = new.idInscripcion And 
            i.GrupalOIndividual = "G") And 
        Exists (Select 1 From esIntegranteDe eid, Inscripcion i 
                Where eid.numCertificado = new.numCertificado 
                And new.idInscripcion != eid.idInscripcion
                And i.idInscripcion = eid.idInscripcion 
                And i.GrupalOIndividual = "G");
    END;
\end{lstlisting}
  
\emph{
  La disciplina de una competencia debe corresponderse a un tipo de categoria particular: combate: peso, edad, sexo, dan.
}

\begin{lstlisting}[language=SQL]
CREATE TRIGGER categoriaCorrectaCombate AFTER INSERT ON Competencia
    BEGIN
        Select Raise(Rollback, "Las categorias de una competencia de "
        "Combate deben ser: Peso, Edad, Genero y Graduacion.")
        From Categoria c
        Where new.tipo = "C" And
        (c.idCategoria not in (Select cP.idCategoria 
        From CategoriaPeso cP)
        Or c.idCategoria not in (Select cP.idCategoria 
        From CategoriaEdad cP)
        Or c.idCategoria not in (Select cP.idCategoria 
        From CategoriaDan cP));
    END;
\end{lstlisting}
  
\emph{
  La disciplina de una competencia debe corresponderse a un tipo de categoria particular: formas: sexo edad, dan.
}

\begin{lstlisting}[language=SQL]
CREATE TRIGGER categoriaCorrectaFormas AFTER INSERT ON Competencia
    BEGIN
        Select Raise(Rollback, "Las categorias de una competencia de "
        "Formas deben ser: Edad, Genero y Graduacion.")
        From Categoria c
        Where new.tipo = "F" And
        (c.idCategoria in (Select cP.idCategoria 
        From CategoriaPeso cP)
        Or c.idCategoria not in (Select cP.idCategoria 
        From CategoriaEdad cP)
        Or c.idCategoria not in (Select cP.idCategoria 
        From CategoriaDan cP));
    END;
\end{lstlisting}
\emph{
  La disciplina de una competencia debe corresponderse a un tipo de categoria particular: salto: sexo, edad, dan.
}

\begin{lstlisting}[language=SQL]
CREATE TRIGGER categoriaCorrectaSalto AFTER INSERT ON Competencia
    BEGIN
        Select Raise(Rollback, "Las categorias de una competencia de "
        "Salto deben ser: Edad, Genero y Graduacion.")
        From Categoria c
        Where new.tipo = "S" And
        (c.idCategoria in (Select cP.idCategoria 
        From CategoriaPeso cP)
        Or c.idCategoria not in (Select cP.idCategoria 
        From CategoriaEdad cP)
        Or c.idCategoria not in (Select cP.idCategoria 
        From CategoriaDan cP));
    END;

\end{lstlisting}
\emph{
  La disciplina de una competencia debe corresponderse a un tipo de categoria particular: rotura: sexo, dan.
}

\begin{lstlisting}[language=SQL]
CREATE TRIGGER categoriaCorrectaRotura AFTER INSERT ON Competencia
    BEGIN
        Select Raise(Rollback, "Las categorias de una competencia de "
        "Rotura deben ser: Genero y Graduacion.")
        From Categoria c
        Where new.tipo = "R" And
        (c.idCategoria in (Select cP.idCategoria 
        From CategoriaPeso cP)
        Or c.idCategoria in (Select cP.idCategoria 
        From CategoriaEdad cP)
        Or c.idCategoria not in (Select cP.idCategoria 
        From CategoriaDan cP));
    END;

\end{lstlisting}
\emph{
}

\begin{lstlisting}[language=SQL]
CREATE TRIGGER categoriaCorrectaCombatePorEquipos 
AFTER INSERT ON Competencia
    BEGIN
        Select Raise(Rollback, "La categoria de una competencia de "
        "Combate por Equipos debe ser por Genero.")
        From Categoria c
        Where new.tipo = "cE" And
        (c.idCategoria in (Select cP.idCategoria 
        From CategoriaPeso cP)
        Or c.idCategoria in (Select cP.idCategoria 
        From CategoriaEdad cP)
        Or c.idCategoria in (Select cP.idCategoria 
        From CategoriaDan cP));
    END;

\end{lstlisting}

\emph{Un participante debe cumplir con las restricciones de la categoría en la que compite}

\begin{lstlisting}[language=SQL]
CREATE TRIGGER cumpleConLaCategoria AFTER INSERT ON esIntegranteDe
    BEGIN
        Select Raise(Rollback, "El participante no cumple "
        "con las categorias de la competencia.")
        From Estudiante e, Participante p, Inscripcion i, 
        esEn es, Competencia comp, Categoria c
        Where e.numCertificado = new.numCertificado 
        And p.numCertificado = new.numCertificado 
        And new.idInscripcion = i.idInscripcion 
        And i.idInscripcion = es.idInscripcion 
        And es.idCompetencia = comp.idCompetencia 
        And comp.idCategoria = c.idCategoria
        And (e.genero != c.genero
        Or Exists (Select 1 From CategoriaPeso cP 
        Where c.idCategoria = cP.idCategoria And 
                (cP.minimo > e.peso Or cP.maximo < e.peso))
        Or Exists (Select 1 From CategoriaEdad cP 
        Where c.idCategoria = cP.idCategoria And 
                (cP.minima > (2017 - p.fechaDeNacimiento) 
                Or cP.maxima < (2017 - p.fechaDeNacimiento)))
        Or Exists (Select 1 From CategoriaDan cP 
        Where c.idCategoria = cP.idCategoria And 
                cP.dan != e.graduacion));
    END;

\end{lstlisting}
\emph{
  Si un arbitro arbitra mas de una competencia, todas tienen que ser en el mismo ring.
}

\begin{lstlisting}[language=SQL]
CREATE TRIGGER unSoloRingPorArbitro AFTER INSERT ON esArbitradaPor
    BEGIN
        Select Raise(Rollback, "Un arbitro no puede "
        "arbtirar en mas de un ring.")
        From esArbitradaPor eAP 
        Where eAP.numDePlacaArbitro = new.numDePlacaArbitro 
        And eap.numeroDeRing != new.numeroDeRing;
    END;

\end{lstlisting}
\emph{
  Un arbitro no puede arbitrar una categoria de mayor graduacion que la suya."
}

\begin{lstlisting}[language=SQL]
CREATE TRIGGER noPuedeArbitrarMayorGraduacion 
AFTER INSERT ON esArbitradaPor
    BEGIN
        Select Raise(Rollback, "Un arbitro no puede arbitrar una "
        "categoria de mayor graduacion que la suya.")
        From Competencia comp, CategoriaDan c, Arbitro a
        Where new.idCompetencia = comp.idCompetencia 
        And a.numDePlaca = new.numDePlacaArbitro 
        And comp.idCategoria = c.idCategoria 
        And a.graduacion <= c.dan;
    END;


\end{lstlisting}
\emph{
  No puede haber competencias del mismo tipo con iguales categorias.
}

\begin{lstlisting}[language=SQL]
CREATE TRIGGER sinModalidadesYCategoriasRepetidas 
AFTER INSERT ON Competencia
    BEGIN
        Select Raise(Rollback, "No puede haber dos competencias de "
        "igual modalidad y con categorias iguales .")
        From Competencia comp, Categoria cNew, Categoria cOtra
        Where new.idCompetencia != comp.idCompetencia 
        And new.tipo = comp.tipo 
        And new.idCategoria = cNew.idCategoria 
        And comp.idCategoria = cOtra.idCategoria 
        And cNew.genero = cOtra.genero
        And Not Exists (Select 1 From CategoriaPeso cPN, 
        CategoriaPeso cPO 
        Where cNew.idCategoria = cPN.idCategoria 
            And cOtra.idCategoria = cPO.idCategoria 
            And cPN.minimo = cPO.minimo 
            And cPN.maximo = cPO.maximo)
        And Not Exists (Select 1 From CategoriaEdad cEN, 
        CategoriaEdad cEO 
        Where cNew.idCategoria = cEN.idCategoria 
            And cOtra.idCategoria = cEO.idCategoria 
            And cEN.minima = cEO.minima And cEN.maxima = cEO.maxima)
        And Not Exists (Select 1 From CategoriaDan cDN, 
        CategoriaDan cDO 
        Where cNew.idCategoria = cDN.idCategoria 
            And cOtra.idCategoria = cDO.idCategoria 
            And cDN.dan = cDO.dan);
    END;
\end{lstlisting}
